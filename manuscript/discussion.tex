\section{Discussion}

\subsection{Extra-Cellular Matrix}

\textit{Focal adhesion}, the adhesion of the cell to the \acrshort{ecm}, is the most deregulated pathway in the \acrshort{pdcl} datasets.
Pathway involved in the collagen biosynthesis, a constituent of the \acrshort{ecm}, are frequently deregulated alongside this pathway.
Evidence in the litterature support this result, showing the role of collagen and cell's interaction in glioblastoma growth, invasiness and migration.
Futhermore, constituent of the matrix in the tumour microenvironment differ from the composition of the matrix in normal brain with higher collagen concentration \cite*{Mammoto2013}.
Studies in the litterature describe different mechanism by which collagen can influence tumour growth.
Collagen concentration and structure (crosslinking) has been documented to influence glioblastoma growth and invasion \cite*{Kaphle2019,Kaufman2005,Rao2013}. 
Glioblastoma adopt a rounded shape in collagen-IV while they adopt a spindle shape in collagen-I/III, showing an effect of the type of collagen on tumour's morphology \cite*{Rao2013}.
Genes associated with fibrillar collagen and coding for collagen processing enzymes are upregulated in glioblastoma compared to grade III gliomas.
As an example, a protein that binds to collagen to internalize it, called Endo180, is found upregulated in grade IV gliomas and correlate with collagen I deposition. 
This protein is lowly expresed in normal brain but is strongly associated with the mesenchymal subclass of gliomas which is associated with grade IV gliomas, thus it has potential as a biomarker \cite*{Huijbers2010}.
Although its expression differ between cell lines, collagen XVI has been found upregulated in glioblastoma compared to normal cells.
Inhibition of endogenous expression of collagen XVI result in reduced cell's adhesion to surface \cite*{Senner2008}.
Collagen has been reported to affect angiogenesis, the formation of new blood vessels, important for tumour growth.
Inhibition of collagen crosslinking and expression reduce angiogenesis and progression of glioblastoma whithout increasing cell invasion, hypoxia or necrosis in mice tumour compared to VEGF targeting anti-angiogenic drugs \cite*{Mammoto2013}.
Thus, targeting collagen structure and expression appears to be a potential therapy to decrease angiogenesis without the adverse effects of regular angiogenesis therapy.
Those evidences combined with the result of our study suggest that targeting \acrshort{ecm} of the glioblastoma microenvironment is viable strategy in therapy.

\subsection{Cholesterol Biosynthesis}

\subsection{Limits of the study}

