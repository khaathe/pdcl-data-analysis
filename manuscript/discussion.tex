\section{Discussion}

\subsection{Extra-Cellular Matrix}

\textit{Focal adhesion}, the adhesion of the cell to the \acrshort{ecm}, is the most deregulated pathway in the \acrshort{pdcl} datasets.
Pathway involved in the collagen biosynthesis, a constituent of the \acrshort{ecm}, are frequently deregulated alongside this pathway.
Evidence in the litterature support this result, showing the role of collagen and cell's interaction in glioblastoma growth, invasiness and migration.
Futhermore, constituent of the matrix in the tumour microenvironment differ from the composition of the matrix in normal brain with higher collagen concentration \cite*{Mammoto2013}.
Studies in the litterature describe different mechanism by which collagen can influence tumour growth.
Collagen concentration and structure (crosslinking) has been documented to influence glioblastoma growth and invasion \cite*{Kaphle2019,Kaufman2005,Rao2013}. 
Glioblastoma adopt a rounded shape in collagen-IV while they adopt a spindle shape in collagen-I/III, showing an effect of the type of collagen on tumour's morphology \cite*{Rao2013}.
Genes associated with fibrillar collagen and coding for collagen processing enzymes are upregulated in glioblastoma compared to grade III gliomas.
As an example, a protein that binds to collagen to internalize it, called Endo180, is found upregulated in grade IV gliomas and correlate with collagen I deposition. 
This protein is lowly expresed in normal brain but is strongly associated with the mesenchymal subclass of gliomas which is associated with grade IV gliomas, thus it has potential as a biomarker \cite*{Huijbers2010}.
Although its expression differ between cell lines, collagen XVI has been found upregulated in glioblastoma compared to normal cells.
Inhibition of endogenous expression of collagen XVI result in reduced cell's adhesion to surface \cite*{Senner2008}.
Collagen has been reported to affect angiogenesis, the formation of new blood vessels, important for tumour growth.
Inhibition of collagen crosslinking and expression reduce angiogenesis and progression of glioblastoma whithout increasing cell invasion, hypoxia or necrosis in mice tumour compared to VEGF targeting anti-angiogenic drugs \cite*{Mammoto2013}.
Thus, targeting collagen structure and expression appears to be a potential therapy to decrease angiogenesis without the adverse effects of regular angiogenesis therapy.
Those evidences combined with the result of our study suggest that targeting \acrshort{ecm} of the glioblastoma microenvironment is viable strategy in therapy.

\subsection{Cholesterol Biosynthesis}

In this study, we found the cholesterol metabolism to be altered in the \acrshort{pdcl} dataset.
Brain is the most cholesterol-rich organ with approximately 20\% of the whole body cholesterol.
Brain cholesterol metabolism differ from other organs as the cholesterol cannot pass through the \acrfull{bbb}.
Hence, cholesterol must be synthetized de novo in the brain \cite*{Villa2016,Yamamoto2018,Pirmoradi2019}.
As for interactions between the cell and the \acrlong{ecm}, few evidences in the litterature describe varied mechanisms by which high level of cholesterol may promote tumour growth.
Firstly, densely packed normal astrocytes inhibit cholestestol synthesis while glioma cells maintain high cholesterol levels.
Glioma cells were sentive to cholesterol synthesis inhibition while astrocytes were not, suggesting that this pathway can be a viable target in therapy \cite*{Kambach2017}.
Studies describe the antitumoral effect of many drugs targeting the cholesterol metabolism in glioblastomas.
For example, \acrfull{lxr} agonist LXR-623 is able to induce cell death in glioblastoma mouse model and spare normal cell by downregulating \acrfull{ldlr} mediated cholesterol intake \cite*{Villa2016, Pirmoradi2019}.
Phytol and retinol exhibit cytoxic activity in glioblastoma cell lines (U87MG, A172 and T98G) in a dose-dependent way by down-regulating genes involved in cholesterol and/or fatty acid synthesis \cite*{Facchini2018}.
Targeting cholesterol traffiking with itraconazole suppresses the growth of glioblastoma by inducing autophagy \cite*{Liu2014} while alkylphospholipids also induced cell cycle arrest in G2/M phase in U87MG cells \cite*{Rios-Marco2013}.
Cholesterol has been observed to modulate the efficiency of \acrfull{tmz} is a drug widely used in treatment for glioblastoma.
Cholesterol can increase the rigidity of the \acrshort{bbb} membrane, as a consequence distribution the active metabolism of \acrshort{tmz} (MTIC) is slowed-down and cannot reach cytotoxic dose in tumoural tissue \cite*{Ramalho2019}.
\acrshort{tmz} apoptotosis activation through the DR5 plasma membrane death receptor is enhanced by higher levels of intracellular cholesterol \cite*{Yamamoto2018}.
The RTK/RAS/PI3K signaling pathway is frequently altered in glioblastoma with its component EGFR upregulated \cite*{McLendon2008}.
EGFR frequent upregulation causes the overexpression of YTHDF2, a downstream effector of the EGFR/SRC/ERK signaling.
As a result, downregulation of the of LXR$\alpha$ inhibit LXR$\alpha$-dependent cholesterol homeostasis which promote glioblastoma proliferation and invasion. 
Taken together, those evidences confirm that cholesterol synthesis is dysregulated in glioblastoma and indicate its potential for therapy.

\subsection{Limits of the study}

