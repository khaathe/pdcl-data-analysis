\begin{abstract}

RNA-Seq technology has become a standard to study gene expressions and discover how diseases alter biological mechanisms.
Despite its wide utilization and range of applications, study and interpretation of RNA-Seq data still remain a challenge for most scientists. 
Due to the variety of protocols described in the literature, it is especially hard for newcomers.
A common approach to interpreting RNA-Seq data is to use Pathway Enrichment analysis, a statistical method that finds enriched pathways in a list of genes of interest.
In other words, this method turns a large list of genes into a smaller list of pathways.
In this paper, we describe a workflow to analyze RNA-Seq data.
We have RNA-Seq data from 20 patients diagnosed with glioblastoma.
We also downloaded RNA-Seq data from the TCGA-GBM project, an initiative to sequence glioblastoma mutation and gene expression, from the \acrfull{tcga} database.
The workflow starts with a \acrfull{de} analysis to determine which genes are deregulated in the case of glioblastoma.
With this step we deduce a list of genes of interest.
We selected the two tools DESeq2 and \acrfull{penda} for their performance and ease of use.
We submit the list of genes to G:Profiler and \acrfull{gsea}, two pathway enrichment methods, to assess which pathways are present in the list.
Because this method determines a typical profile of deregulations for glioblastoma, we performed an analysis where we compare each sample of the \acrshort{pdcl} dataset one by one to all controls.
This second analysis better accounts for heterogeneity among the samples.
Lastly, we select the pathways that pass the significance threshold in both G:Profiler and \acrshort{gsea}.
We compare the result between \acrshort{tcga} and \acrshort{pdcl}.
We found that most of the mechanisms that are found altered in both \acrshort{tcga} and \acrshort{pdcl} datasets are involved in the Cell-Cycle, DNA replication and repair.
In the \acrshort{pdcl} dataset, the most commonly deregulated pathway is the \textit{focal adhesion} of the cell to the matrix.
\textit{Collagen formation} and \textit{cholesterol synthesis} are pathways frequently deregulated as well. 
A review of the literature shows that collagen is involved in glioblastoma growth, invasion and migration while cholesterol has been linked to glioblastoma growth.
In addition, cholesterol synthesis has been suggested as a potential drug target for therapy ignoring healthy cells.

\end{abstract}